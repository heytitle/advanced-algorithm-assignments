% !TEX root = ../main.tex

\tikzset
{
    treenode/.style = {circle, draw=black, align=center, minimum size=1cm},
}

\section*{IO.III-1}

\subsection*{(i)}
When the subarray fits into 1 block, then one of the subtree does not need any more I/Os and the other one needs one more I/O (because the next center is in a consecutive block).\\ 

The number of steps needed to reach such case is $\log(n/B)$. Then the error of the number of I/Os is $O(1)$. We have:
$$\#I/Os = \log(n/B) \pm O(1)$$

\subsection*{(ii)}
Instead of storing blocks as in-order traversal, we will separate the tree in to small subtrees of size B and store it into a block as shown in figure \ref{fig:blocking}.
We know that the depth of the tree is $\log n$ and the depth of subtree is $\log B$. Therefore IO's complexity of root-to-leaf traversal is :
$$T_{io}(n) = \frac{\log n}{\log B} = \log_{B} n$$

As shown in figure \ref{fig:blocking}, this blocking strategy make the tree look similar to B-tree in a sense that
its nodes are grouped into subtrees and each subtree is comparable to a node in B-tree that can have more than 1 element.

\begin{figure}[t]
%\centering
{
    \begin{tikzpicture}[->,>=stealth', level/.style={sibling distance = 10cm/#1, level distance = 1.5cm}, scale=0.6,transform shape]
    \node [treenode] {$8$ \\ B1}
    child
    {
        node [treenode] {$4$ \\ B1} 
        child
        {
            node [treenode] {$2$ \\ B2} 
            child
            {
            		node [treenode] {$1$ \\ B2 }
            }
            child
            {
            		node [treenode] {$3$ \\ B2}
            }
        }
        child
        {
            node [treenode] {$6$ \\ B3} 
                        child
            {
            		node [treenode] {$5$ \\ B3 }
            }
            child
            {
            		node [treenode] {$7$ \\ B3}
            }
        }
    }
    child
    {
        node [treenode] {$12$ \\ B1} 
        child
        {
            node [treenode] {$10$ \\ B4} 
            child
            {
            		node [treenode] {$9$ \\ B4 }
            }
            child
            {
            		node [treenode] {$11$ \\ B4}
            }
        }
        child
        {
            node [treenode] {$14$ \\ B5} 
                        child
            {
            		node [treenode] {$13$ \\ B5 }
            }
            child
            {
            		node [treenode] {$15$ \\ B5}
            }
        }
    }
    ;
    \end{tikzpicture}
    \caption{An illustration of subtree blocking strategy when B=3 \label{fig:simple}}
    \label{fig:blocking}
}
\end{figure}
