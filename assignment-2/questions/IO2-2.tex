\section*{IO.II-2}
\subsection*{(i)}
Let $n$ be the total number of items to be sorted, and $k = M/B - 1$ be the number of subarrays.\\

The complexity of the EM-MergeSort is the cost of the recurrence plus the cost of merging $k$ subarrays of size $n/k$ into an array of size $n$. \\ 

For each element, we have to perform $k-1$ comparisons to put it to the proper position in the final array. We have $n$ element in total, so the cost of merging is $O(n(k - 1)) = O(nk)$.\\

We can write the recurrence formula of the EM-MergeSort as follows:

\begin{align*}
	T(n) = \begin{cases}
	    1& \text{if } n = 1\\
        2T(n/k)+O(nk)              & \text{otherwise}
	\end{cases}
\end{align*}

The base case is when $n = 1$, then $T(n) = 1$.\\

Analyzing this recurrence formula, we have:
\begin{align*}
    T(n) &= k^2T(n/(k^2)) + 2O(nk)\\
    &= k^3T(n/(k^3)) + 3O(nk) \\
    &= \cdots \\
    &= k^\alpha T(n/(k^\alpha)) + \alpha O(nk)
\end{align*}

$T(n)$ reaches the base case when $n/(k^\alpha) = 1 \iff \alpha = \log_k(n)$, so the final formula is:

\begin{align*}
    T(n) &= k^{log_k(n)} (n/(k^{log_k(n)})) + \log_k(n)O(nk) \\
    &= n + \log_k(n)O(nk)\\
    &= O(nk\log_k(n))
\end{align*}

\subsection*{(ii)}
We propose another merging strategy as shown in algorithm \ref{alg:merge}.

\begin{algorithm}[h]
  \caption{Merging k arrays}
  \label{alg:merge}
  \begin{algorithmic}
      \Require List of subarrays $S = [A_1, A_2, \cdots, A_k]$
    \Ensure Merged Array $A$

    \If{ $( k \mod 2 ) > 0$} 
  \State $A_{last} = A{k}$
  \State $k = k - 1$
  \EndIf
  \While{$SizeOf(S) > 1$}
  \State Construct $k/2$ pairs of subarrays $\{A_1, A_2\},\cdots, \{A_{k - 1}, A_k\}$
  \State S := [\,\,]
  \For{Each pair $\{A_i, A_j\}$}
  \State Merge $A_i$ and $A_j$ into $A_{ij}$
  \State Add $A_{ij}$ to S
  \EndFor
  \EndWhile

  \If{$A_{last}$ NOT NULL}
  \State Merge S[0] with $A_{last}$
  \EndIf

  \Return S[0]
\end{algorithmic}
\end{algorithm}

Now we analyze the complexity of the new merging strategy, and then the complexity of the EM-MergeSort using this as the main merging routine. \\

For each step, we reduce the number of pairs by 2. So it takes $O(\log(k))$ steps to converge to 1 array.\\ 

For each pair of size $n/k$, it takes $O(2n/k)$ comparisons to merge them into 1. In each step of the merge, we perform $O(n)$ comparisons.\\

In general, the complexity of the merge step is $O(n\log(k))$. So the recurrence formula of the K-way MergeSort is:

\begin{align*}
	T(n) = \begin{cases}
	    1& \text{if } n = 1\\
        2T(n/k)+O(n\log{k})              & \text{otherwise}
	\end{cases}
\end{align*}

We have:

\begin{align*}
    T(n) &= k^2 T(n/(k^2)) + 2O(n\log(k)) \\
    &= k^3 T(n/(k^3)) + 3O(n\log(k)) \\
    &= \cdots \\
    &= k^\alpha T(n/(k^\alpha)) + \alpha O(n\log(k))
\end{align*}

$T(n)$ reaches the base case when the problem size reaches 1. It takes $\log_k n$ steps to do so, then:

\begin{align*}
T(n) &= n + \log_k(n) O(n\log(k)) \\
        &= n + O(n\frac{log(n)\log(k)}{\log(k)}) \\
        &= O(n\log(n))
\end{align*}

So our new merging strategy satisfies the condition.

