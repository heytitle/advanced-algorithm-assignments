\section*{IO.I-2}
\subsection*{(i)}
Let $T(n)$ denote IO's of the problem and $T_x$, $T_y$ and $T_z$ denote IO's of matrix $X$, $Y$ and $T_z$ respectively.
For each cell in $Z$, we need
\begin{align*}
	T_x &\leq \frac{\sqrt{n}+2(B-1)}{B}\\
	T_y &\leq \sqrt{n} \\
	T_z &= 1
\end{align*}
Then, the total IO's we need for computing a cell is :
$$\frac{\sqrt{n}+2}{B} + \sqrt{n} + 1$$
Therefore, the total IO's that we need to compute the product $Z=XY$ :
\begin{align*}
	T(n) &\leq ( \frac{\sqrt{n}+2(B-1)}{B} + \sqrt{n} + 1	)n\\
	&\leq \frac{n\sqrt{n}+2n(B-1)}{B} + n\sqrt{n} + n \\
	&= O(n\sqrt{n})
\end{align*}

\subsection*{(ii)}
If $Y$ is stored in $column-major$ order. For each cell, we will use $$T_y=\frac{\sqrt{n}+2(B-1)}{B}$$.
Therefore, the total IO's is
\begin{align*}
	T(n) &\leq ( 2\frac{\sqrt{n}+2(B-1)}{B} + 1	)n\\
	&\leq \frac{2n\sqrt{n}+4n(B-1)}{B} + n \\
	&= O(\frac{n\sqrt{n}}{B})
\end{align*}

\subsection*{(iii)}

% Image of X and Y how to find memory needed.

In order to compute sub-problem, all variables that we have in sub-problem should fit into main memory. 

Let $M_x$, $M_y$ and $M_z$ denote memory space that is required by $X$, $Y$ and $Z$ when computing a sub-problem

For each sub problem, we need
\begin{align*}
	M_x &= t(2t + 2(B-1)) \\
	M_y &= 2t(t+2(B-1))\\
	M_z &= t(t+2(B-1))
\end{align*}
Let $M$ denote the total memory we have. Then, we find IO's recursive base case which happens when all variables in sub-problem can fit into main memory.
\begin{align*}
M_x + M_y + M_z &\leq M \\
5n^2 + 6(B-1)n &\leq M
\end{align*}
Thus, we can formulate recurrence IO's complexity function of this algorithm.
\begin{align*}
	T(t) = \begin{cases}
	    \frac{5t^2 + 6(B-1)t}{B},& \text{if } 5t^2 + 6(B-1)t \leq M\\
	    4T(\frac{t}{2}),              & \text{otherwise}
	\end{cases}
\end{align*}
Hence, we have a general form of the function where $k$ is a depth of the recursive call.
\begin{align*}
	T(t) = 4^{k}T(\frac{t}{2^{k}})
\end{align*}
We know that if $\frac{n}{2^{k}} = 5n^2 + 6(B-1)n$ we will reach the base case. Then, we can find $k$.
\begin{align*}
	\frac{t}{2^k} &= 5t^2 + 6(B-1)t \\
	\frac{1}{2^k	} &= 5t + 6(B-1) \\
	2^k &= \frac{1}{5t + 6(B-1)} \\
	k\log2 &= \frac{1}{5t + 6(B-1)} \\
	k &= \frac{1}{5t + 6(B-1)}
\end{align*}

Before we derive $T(t)$, we will denote $A$ as
$$\log_2 \frac{1}{5t + 6(B-1)}$$

Next, we can simply derive the IO's complexity of the algorithm.
\begin{align*}
	T(t) &\leq 4^{k}T(\frac{t}{2^{k}}) \\
	&\leq 4^{log_2 A}T\Big(\frac{t}{2^{log_2 A}}\Big) \\
	&\leq A^2T(\frac{t}{A}) \\
	&\leq A^2\Big( \frac{5\frac{t}{A}^2}{B} + \frac{6(B-1)\frac{t}{A}}{B} \Big) \\
	&\leq \frac{5t^2}{B} + \frac{6(B-1)t}{AB} \\
	&= O(\frac{t^2}{B})
\end{align*}

Therefore the IO's complexity of the algorithm is $O(t^2/B)$.