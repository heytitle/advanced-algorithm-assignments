% !TEX root = ../main.tex
\section*{IO.III-3}

\begin{algorithm}
  \caption{Coloring Undirected Graph}
  \label{alg:global_minimum}
  \begin{algorithmic}
    \renewcommand{\algorithmicrequire}{\textbf{Input:}}
    \renewcommand{\algorithmicensure}{\textbf{Output:}}
    \algnewcommand\algorithmicoperation{\textbf{Operation:}}
    \algnewcommand\Operation{\item[\algorithmicoperation]}
    \Require $G( V, E )$
    \Ensure Color of $v_i \in V$
    \Operation
    \State Convert $G$ into a DAG graph $G^*$ by directing every edge ( $v_i$, $v_j$ ) where $i < j$
    \State Initialise an array F[0..$n-1$], a min-priority queue $Q$
    \State Initialise a set of colour C[0..$d_{max}$]
    \For {$i$ := 0 to $n-1$}
    	\State Performs $Extract-Min$ operations on $Q$ until getting a pair whose priority $i' > i$
	\State Insert the extra extracted pair back to $Q$
	\State $f(v_i)$ := Choose one colour from C $\setminus$ \{ extracted pairs \}
	\State F[i] := $f(v_i)$
	    \For { each $j$ in $V[i].OutNeighbors$ }
	        \State Insert a pair ( $f(v_i)$, $j$ ) into $Q$
	   \EndFor
    \EndFor \\
    \Return F
  \end{algorithmic}
\end{algorithm}
\textbf{Theorem.} The algorithm performs $O( SORT(|V|+|E|) )$ and returns valid solution.

\begin{proof}
    Let denote $N_{in}$ a set of in-neighbor of $v_i \in V$. We observe that each iteration the algorithm performs
    $$2O( SORT(1 + |N_{in}(v_i)| ))$$
One from $Extract-Min$ operation and the other from $Insertions$. \\
Thus, the total I/O's complexity  is :
\begin{align*}
    T_{io}(G(V,E)) &= \sum^{n-1}_{i=0}{O( 1 + |N_{in}|)(v_i)} \\
    &= O( SORT( |V|+|E| ))
\end{align*}

To prove that the algorithm returns valid solution. We first argue that the algorithm does not use more than $d_{max} + 1 $ colours .
Secondly, there are no 2 adjacent nodes having same colour because each time it picks a colour the list of colours is subtracted by colours used for its in-neighbors.

\end{proof}
