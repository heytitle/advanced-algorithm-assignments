\section*{AIII-1-i)}
\label{approx-3-1-i}

We assume total time $ T = \sum_{j=1}^{n} t_i $ to be the total time of all the jobs.
Since we define large job as having time  $ t \ge \epsilon T $,
we can deduce that if all runing jobs are large jobs, the maximal number of those jobs is
equal to :
$$ n_{max} = \frac{T}{\epsilon T} = \frac{1}{\epsilon} $$

For each job we take into account there are two possible ways of assigning it to a machine.
Therefore the possible ways jobs can be scheduled is:

$$ 2*2* ... *2 = 2^{\frac{1}{\epsilon}} $$

Since we don't distuingish between the machines, we remove the duplicates leaving the 
total number of schedules at :
$$ \frac{2^\frac{1}{\epsilon}}{2} = 2^{\frac{1}{\epsilon}-1} $$
