\section* {AIII.3}
\label {a3-3}
\subsection*{(i)}
Because we know that $ALG(G,\epsilon) \in \mathbb{N}$, so that if we can find such $\epsilon$ that the algorithm yields
\begin{align*}
	OPT - 1 < ALG\left(G,\epsilon \right) \leq OPT
\end{align*}

Then, we can get $OPT$ in polynomial time.\\

In order to get such $\epsilon$, we will derive
\begin{align*}
	ALG\left(G,\epsilon\right) &> OPT - 1\\
	&> \left( 1 - \frac{1}{OPT} \right)OPT
\end{align*}
%\begin{align*}
%	ALG OPT - 1 \\
%%	&\ge \left( 1 - \frac{1}{OPT} \right) OPT
%	
%\end{align*}
Hence we can get $OPT$ if we choose $\epsilon < \frac{1}{OPT}$ and we also know that the algorithm uses $ALG(G,\epsilon)$ as a subroutine. \\ \\

Therefore, there is no such FPTAS exist.

\subsection*{(ii)}

The proof above indeed implies that there is no PTAS such a problem because we know that a PTAS algorithm also computes a $(1 - \epsilon)$-approximation for the problem and if we choose $\epsilon > \frac{1}{OPT}$ as the proof above then, the PTAS algorithm will yield $OPT$ in polynomial time of $n$. \\

Therefore there is no PTAS exist anymore.


	

