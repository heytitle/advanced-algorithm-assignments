\section* {AIII.3}
\label {a3-3}
\subsection*{(i)}
Suppose that we have such $ALG(G,\epsilon)$ that can return a $(1 - \epsilon )-approximation$ solution to the problem. We can define an FPTAS as follow:
\begin{itemize}
	\item Set $\epsilon$ to some value
	\item Return $ALG(G,\epsilon)$
\end{itemize}

The above algorithm runs in polynomial time. \\

Because we know that $ALG(G,\epsilon) \in \mathbb{N}$, so that if we can find such $\epsilon$ that the algorithm yields
\begin{align*}
	OPT - 1 < ALG\left(G,\epsilon \right) \leq OPT
\end{align*}

Then, we can get $OPT$ in polynomial time.\\

In order to get such $\epsilon$, we need:
\begin{align*}
	ALG\left(G,\epsilon\right) &> OPT - 1\\
	&> \left( 1 - \frac{1}{OPT} \right)OPT
\end{align*}
%\begin{align*}
%	ALG OPT - 1 \\
%%	&\ge \left( 1 - \frac{1}{OPT} \right) OPT
%	
%\end{align*}

This holds when $\epsilon < \frac{1}{OPT}$.\\

Indeed, we do not know the exact value of $OPT$. But the above property implies that there exists some values of $\epsilon$ that can help $ALG(G,\epsilon)$ return $OPT$. Because our algorithm uses $ALG(G,\epsilon)$ as the main routine, so it also runs in polynomial time.

Now we have an algorithm that can return the optimal solution in polynomial time. Then our problem is not NP-Hard anymore. This contradicts with the definition.

Therefore, there is no FPTAS for the problem.

\subsection*{(ii)}

The proof above indeed implies that there exists some values of $\epsilon$ that can help any $(1-\epsilon)-approx$ algorithm return $OPT$.\\

Therefore, if there exists a PTAS for this problem, it also returns $OPT$ with such $\epsilon$ because PTAS also produces $(1-\epsilon)-approx$ solutions. This conflicts with the NP-Hard property of the problem as being explained above.\\

Therefore there is no PTAS for the problem.


	

