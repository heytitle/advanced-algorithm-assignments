\section*{A.I-2}
\label{sec:approx-1-2}

From the question, we know that
\begin{align*}
m &= 10\\
\sum\limits_{j=1}^n t_j &\ge 1000 \\
t_j  &\in [1,20] \,; for \, all \, i \le j \le n 
\end{align*}

Let $T^{\prime}_{i^*}$ denote the load of $M_i$ before $t_j^*$, last job, is assigned to the machine. \\
Thus $T_i^*$, which represents makespan of the assignment, equals to

\begin{align*}
T_i^* = T^{\prime}_{i^*} + t_j^*
\end{align*}

Because $T^{\prime}_{i^*}$ is the minimum load among all machines, it is less than the average of machine loads (excluding $t^*_j$).

\begin{align*}
T^{\prime}_{i^*} \le \frac{1}{m} \bigg[ \sum\limits_{j=1}^n t_j - t_j^*  \bigg] 
\end{align*}

Then we can derive
\begin{align*}
T_i^* &= T^{\prime}_{i^*} + t_j^* \\
&\le \frac{1}{m} \bigg[ \sum\limits_{j=1}^n t_j - t_j^*  \bigg] + t_j^* \\
&\le \frac{1}{m} \sum\limits_{j=1}^n t_j + ( 1 - \frac{1}{m} ) t_j^* \\
&\le 100 + ( 1 - \frac{1}{10} ) 20 \\
&\le 118
\end{align*}

According $Algorithm\, Greedy\, Scheduling$ and the question, we know
\begin{align*}
max \bigg( \frac{1}{m}\sum\limits_{j=1}^n t_j \, , \underset{1\le j \le n }{max(t_j) }\bigg) &\le LB \le OPT \\
\underset{1\le j \le n }{max(t_j)} &= 20 \\
\end{align*}

Then we can derive
\begin{align*}
max \bigg( \frac{1}{m}\sum\limits_{j=1}^n t_j \, , 20 \bigg) &\le LB
\end{align*}

Since $\frac{1}{m}\sum\limits_{j=1}^n t_j \ge 1000$, thus
\begin{align*}
100 &\le LB \\
&\le OPT
\end{align*}

Therefore, approximation-ratio($\rho$) equals to
\begin{align*}
T_i^* &\le \rho \, OPT \\
\frac{118}{100} &\le \rho \\
1.18 &\le \rho 
\end{align*}

For this particular setting, $Algorithm \,Greedy \,Scheduling$ is 1.18 approximation algorithm.

