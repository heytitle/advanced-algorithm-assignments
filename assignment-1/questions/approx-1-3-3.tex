\section*{A.I.3.iii}
\label{a-1-3-3}

The idea of our algorithm is that, we put all the points in to a coordinate system, then we divide the coordinate system into a set of unit rows (i.e. rows with height 1). For each row, we use algorithm \ref{alg:row_minimum} to find the minimum size square-cover. The global min-square-cover is the sum of all row-square-cover.

\begin{algorithm}
  \caption{Finding global minimum square cover}
  \label{alg:global_minimum}
  \begin{algorithmic}
    \renewcommand{\algorithmicrequire}{\textbf{Input:}}
    \renewcommand{\algorithmicensure}{\textbf{Output:}}
    \algnewcommand\algorithmicoperation{\textbf{Operation:}}
    \algnewcommand\Operation{\item[\algorithmicoperation]}
    \Require Set of points $P$
    \Ensure Min Square Cover $min$
    \Operation
    \State currentMin = 0;
    \ForAll {Row $r$ in the space}
    \State currentMin += FindRowMinSquare()
    \EndFor
    \State set $min = currentMin$
    
    \Return min
  \end{algorithmic}
\end{algorithm}

\textbf{Theorem.} FindingGlobalMinimumSC is $2-approximation$

\begin{proof}
  We prove this Theorem by induction.
  
  If the optimal solution consists of only 1 square, then $OPT_1 = 1$. After applying our algorithm, the square can be split into at most 2 squares.

  This is true because if the algorithm returns more than 2 squares, then there is a row which consists of more than 1 square. It means the margin of our points is larger than 1, then there must be more than 1 squares to fit them. It contradicts with our assumption.
  
So $ALG_1 \leq 2 = 2OPT_1$

Suppose when $n = k$, the algorithm is true, that is $ALG_k \leq 2OPT_k$

Now we add some addtional points which insert another square into the optimal solution. Now $n = k + 1$.

We have $OPT_{k+1} = OPT_k + 1$

Applying our algorithm, the final result is the sum of the original input ($n = k$) and the new input ($n = 1$). We know that the Theorem holds for both of them. We have:

\begin{equation}
  ALG_{k + 1} = ALG_k + ALG_1 \leq 2OPT_k + 2 = 2(OPT_k + 1) = 2OPT_{k+1}
\end{equation}

Thus the Theorem also holds for $n = k + 1$. Therefore, it holds for all values of $n$.

  In conclusion, this algorithm is $2-approximation$.

\end{proof}
