\section* {AII.2}
\label {a2-2}
\subsection*{(i)}

Considering the input we can define the best possible scenario for the solution as being a CNF formula where a single element picked from the first clause can be found in all other clasues. The algorithm then would remove all other clauses. This leads us to the assumption that the optimal value $ OPT $ in our case is equal to $ OPT = 1 $ 

Since no assigned method is specified as to which element of the clause we should choose, we assume a random pick. This leads us to the worst possible choice scenario of a variable existing in all the clauses of the CNF and algorithm always picking the wrong one for the search.   

$$ (x_1 \vee x_4  \vee x_5) \wedge (x_3 \vee x_1 \vee x_6) \wedge (x_2 \vee x_1 \vee x_7) $$

If the algorithm picked in the first clause $ x_4 $ as the variable to look for and  $ x_3 $ too look for in the second clause  it would return a value of one element in each clause until the last clause leaving us with an $ ALG = \frac{n}{3}-1 $ . Since all elements are different but they share one element in all clauses we pick 1 element in each clause meaning $ \frac{1}{3} $ of all used elements with exclusion of the first clause. 
This would however vary greatly from the most optimal solution of $ OPT = 1 $ as described previously. 

That leads to the approximation ratio of :
$$ 1 \le \rho \le \frac{n}{3} -1 $$ 


