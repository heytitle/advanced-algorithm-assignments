\subsection*{(ii)}
\label{approx-3-1-ii}


We have 

\begin{align*}
	OPT \geq max(\frac{T}{2}, t_{max})
\end{align*}

where $t_{max}$ is the maximum job size.


To obtain PTAS, We split the jobs up into two types :
\begin{align*}
    \text{Job is } \begin{cases}
        \text{Large} \text{ if } t_j \ge \epsilon T \\
        \text{Small} \text{\, otherwise}
    \end{cases}
\end{align*}


The PTAS algorithm runs as follow.

\begin{algorithm}[H]
  \caption{Load Balancing PTAS}
  \label{alg:load_balancing_ptas}
  \begin{algorithmic}
    \Require Set $J$ of $n$ jobs with running time $t_j$, $\epsilon$
    \Ensure Minimum max(makespan) between two machines.
    \renewcommand{\algorithmicrequire}{\textbf{Input:}}
    \renewcommand{\algorithmicensure}{\textbf{Output:}}
    \algnewcommand\algorithmicoperation{\textbf{Operation:}}
    \algnewcommand\Operation{\item[\algorithmicoperation]}
    \Operation
    \State Define set of large jobs $L$ where $t_j \geq \epsilon T \, \forall t_j \in L$
    \State Define set of small jobs $S := J \setminus L$
    \State List all possible combinations of $t_j \in L$ into 2 machines
    \State Select the combination that has the lowest max(makespan)
    \ForAll {$t_j$ in $S$}
    \State Schedule $ t_j $ to the machine which has the lower current makespan
    \EndFor\\
    \Return The max(makespan) between 2 machines
  \end{algorithmic}
\end{algorithm}

\textbf{Proof}

Let denote:
\begin{itemize}
	\item $ ALG $ as a makespan generated by our solution
	\item $ T' $ as the makespan in the selected machine before assigning the last job
	\item $ t_{last} $ as the size of the last scheduled job
\end{itemize}
We come up with:
\begin{align*}
	ALG &= T' + t_{last} \\
 &\le \frac{T - t_{last}}{2} + t_{last} \\
 &\le \frac{T}{2} + \frac{\epsilon T}{2} \\
 &\le (1 + \epsilon)\frac{T}{2} \\
 &\le (1 + \epsilon )LB \\
 &\le (1 + \epsilon)OPT
\end{align*}

When $\epsilon$ is small enough, $ALG$ will return $OPT$ solution due to the brute-force approach for large jobs. Then the approximation ratio still holds as $(1 + \epsilon)$ where $\epsilon$ is very small ($\epsilon \to 0$).

\textbf{Running Time}
Since there can be only at most $ 2^{\frac{1}{\epsilon}-1} $ possible large jobs, then the brute-force part of the algorithm will have a running time of $ O(2^{\frac{1}{\epsilon}-1}) $. The greedy scheduling of the large jobs is faster with linear time $ O(n) $ for $ n $ jobs. Leaving a total running time - $ O(2^{\frac{1}{\epsilon}-1} + n) $, which is polynomial in term of $n$. So this algorithm satisfies the condition of a PTAS.

