% !TEX root = ../main.tex

\section*{Str.I-3}

\begin{algorithm}[h]
  \caption{ $\epsilon$-Frequent Items}
  \label{alg:frequent-items}
  \begin{algorithmic}
      \Require A stream $\langle a_1, \dotsc, a_m \rangle $ in Cash Register Model where $a_i = (j,c)$
      \State Initialize $I \gets \emptyset$
      \State Process( $a_i $ ):
      \If {  $j \in I $ } 
      	\State $c(j) \gets c(j) + c$
      \Else
      \State Insert $j$ into $I$ with counter $c(j)$ = c
      	\If { $|I| \ge 1/\epsilon$ }
		\State $MinFreq \gets   \min\limits_{  j \in I}  c(j)$
		\For{ all items $j \in I$}
			\State $c(j) \gets c(j) - MinFreq$ ; delete $j$ from $I$ when $c(j)=0$
		\EndFor
	\EndIf
     \EndIf
     
  \Return $I$
\end{algorithmic}
\end{algorithm}

The algorithm in vanilla model can be adapted to cash-register model by initialise $c(j)$, a counter of $j$, with $c$ or increase $c(j)$ by $c$ if $j \in I$. Secondly, if there are too many items in $I$, $ |I| \ge 1/\epsilon$, $c(j)$ for all $j$ in $I$ will be decreased by  $\min\limits_{  j \in I}  c(j)$.

Next, let denote a token $( j^*, c^* )$ as a token that is going to be processed next and $j_{min}$ whose $c(j_{min})  = \min\limits_{  j \in I}  c(j)$ in that time.  We will argue that the algorithm in cash-register model correctly computes a superset of the $\epsilon$-frequent items by comparing its behaviours with the original algorithm in vanilla model in 2 conditions, namely when $c^* \ge minFreq$ and $c^* < minFreq$.

For $c^* \ge minFreq$, after the original algorithm processes a token $(j^*, 1)$ $c(j_{min})$ time, $j_{min}$ will be removed from $I$ and when it processes the rest of $(j^*, 1)$, $c(j^*)$ will equal to $c^* - c(j_{min})$. This is exactly the same as the modified algorithm does when $(j^*, c^*)$ arrives, $j_{min}$ will be removed since $c(j_{min})$ becomes zero after subtracted by itself and $j^*$ remains in $I$ with $c(j^*) := c(j^*) - c(j_{min})$.

For $c^* < minFreq$, after the original algorithm processes a token $(j^*, 1)$  $c^*$ time,  $c(j)$ for  $j \in I$ will be subtracted by $c^*$ and there is no $j^*$ in $I$ which is the same result from the modified algorithm does.

Therefore, we can conclude that the modified algorithm returns correct result.
