\section*{II-1}
\subsection*{(i)}
We know that $\expt{\widetilde{\Phi}(\sigma)} = \Phi(\sigma)$ and $\var{\widetilde{\Phi}(\sigma)} = (1/3)\cdot \Phi(\sigma)$, thus:

$$\pr{|\widetilde{\Phi}(\sigma)- \Phi(\sigma)| \geq c\cdot \Phi(\sigma)}
= \pr{|\widetilde{\Phi}(\sigma)- \expt{\widetilde{\Phi}(\sigma)} \geq \
3c \cdot \var{\widetilde{\Phi}(\sigma)}}$$

According to Chebyshev Inequality, we have:

\begin{equation*}
  \begin{aligned}
\pr{|\widetilde{\Phi}(\sigma)- \expt{\widetilde{\Phi}(\sigma)|} &\geq 
3c \cdot \var{\widetilde{\Phi}(\sigma)}} \\
= \pr{|\widetilde{\Phi}(\sigma)- \expt{\widetilde{\Phi}(\sigma)|} 
&\geq 3c \cdot \sqrt(\var{\widetilde{\Phi}(\sigma)}) \cdot \sqrt(\var{\widetilde{\Phi}(\sigma)})} \\
&\leq \frac{1}{9 \cdot c^2 \cdot
  \var{\widetilde{\Phi}(\sigma)}}
\end{aligned}
\end{equation*}

To let 1/6 be the upper bound of this probability, we have to choose a value of $c$ such that:

\begin{equation*}
  \begin{aligned}
9 \cdot c^2 \cdot \var{\widetilde{\Phi}(\sigma)} &= 6\\
c &= \sqrt(\frac{6}{9\var{\widetilde{\Phi}(\sigma)}})\\
&= \sqrt(\frac{2}{3 \frac{\Phi(\sigma)}{3}})
\end{aligned}
\end{equation*}

Since we know that $\Phi(\sigma) > 3$, then:

\begin{equation*}
  \begin{aligned}
c &= \sqrt(\frac{2}{\Phi(\sigma)}))
& < \sqrt(\frac{2}{3})
\end{aligned}
\end{equation*}

This value of c statisfies the condition that $0 < c < 1$.

\subsection*{(ii)}

The idea of the new algorithm is to run \alg $k$ times, then taking the median of the $k$ return values $\widetilde{\Phi}(\sigma)$ (the Median trick). Before modifying the algorithm formally, we make some analysis as follows. \\

Let $X_i$ be an indicator random variable, which is defined as:

\[ 
  X_i = \begin{cases}
    1 \text{	if} \pr{|\widetilde{\Phi}(\sigma)- \Phi(\sigma)| \geq c\cdot \Phi(\sigma)} \\
    0 \text{	otherwise}
  \end{cases}
\]

Let $X = \sum_{i=1}^k X_i$ be the random variable representing the final outcome of the new algorithm. We have:

\begin{equation*}
  \expt{X} = \expt{\sum_{i=1}^k X_i} = \sum_{i=1}^k \expt{X_i}
\end{equation*}

In this algorithm, we used the specified value of $c$ in part (i), which confirms that:

\begin{equation*}
  \begin{aligned}
&\pr{|\widetilde{\Phi}(\sigma)- \Phi(\sigma)| \geq c\cdot \Phi(\sigma)} \leq 1/6 \\ \\
&\text{which implies that} \\ 
&\pr{X_i = 1} \leq 1/6 \\ \\
&\text{Because $X_i$ is an indicator random variable, then:}\\
&\expt{X_i} = 1/6 \\ \\
&\text{Thus, we have:} \\
&\expt{X} = \sum_{i=1}^k \expt{X_i} = k / 6 \\
\end{aligned}
\end{equation*}

When $X_i = 1$, it implies that we do not have the event described in the problem statement. The probability that we do not have it after performing the Median trick is:

\begin{equation*}
  \begin{aligned}
    \pr{X > \frac{k}{2}} &= \pr{X > 3\expt{X}} = \pr{X > (1 + 2)\expt{X}} \\
    &\leq (\frac{e^2}{3^3})^{\frac{k}{6}} \\
    &= (\frac{e^2}{27})^{\frac{k}{6}} \\
  \end{aligned}
\end{equation*}

Thus, the probability that we get the event described is $1 - (\frac{e^2}{27})^{\frac{k}{6}}$. To get the desired value ($1 - \delta$), we have to choose $k$ such that:

\begin{equation*}
  \begin{aligned}
    \delta &\geq (\frac{e^2}{27})^{\frac{k}{6}} \\
    \iff \frac{1}{\delta} &\leq (\frac{27}{e^2})^{\frac{k}{6}} \\
    \iff \log \frac{1}{\delta} &\leq \frac{k}{6} \log \frac{27}{e^2} \\
    \iff k &\geq \frac{6}{\log \frac{27}{e^2}} \cdot \log \frac{1}{\delta} \\
    \text{We can choose} \\
    k &= \lceil 4\log \frac{1}{\delta}\rceil \\
\end{aligned}
\end{equation*}

In conclusion, we modify the algorithm as represented in algorithm \ref{alg:revise}.

\begin{algorithm}
  \caption{Revised Algorithm Taking \alg as Sub-Routine}
  \label{alg:revise}
  \begin{algorithmic}
    \Require A stream $\sigma = <a_1, \cdots, a_m>, \delta$
    \Ensure Statistics $\widehat{\Phi}(\sigma)$
    \renewcommand{\algorithmicrequire}{\textbf{Input:}}
    \renewcommand{\algorithmicensure}{\textbf{Output:}}
    \algnewcommand\algorithmicoperation{\textbf{Operation:}}
    \algnewcommand\Operation{\item[\algorithmicoperation]}
    \Operation
    \State Choose $k := \lceil 4\log \frac{1}{\delta}\rceil$
    \State Init $J := \emptyset$
    \For {i from 1 to k}
    \State $\widetilde{\Phi}_i(\sigma) :=$ \alg()
    \State $J = J \cup \widetilde{\Phi}_i(\sigma)$ 
    \EndFor

    \Return The median of set $J$
  \end{algorithmic}
\end{algorithm}

The algorithm needs to store the set $J$ of $k$ elements, and each run of \alg stores $B(n,m)$ bits, so the total number of bits used by the algorithm is $O(kB(n,m))$.
