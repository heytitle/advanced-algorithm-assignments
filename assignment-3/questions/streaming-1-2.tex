% !TEX root = ../main.tex
\algdef{SE}[SUBALG]{Indent}{EndIndent}{}{\algorithmicend\ }%
\algtext*{Indent}
\algtext*{EndIndent}

\section*{Str.I-2}

\begin{algorithm}[h]
  \caption{Find Two Missing Item}
  \label{alg:find-2-missing-item}
  \begin{algorithmic}
      \Require A stream $\langle a_1, \dotsc, a_{n-2} \rangle$ where $a_i = (j,1)$
      \State Compute $remainingSum \gets n(n+1)/2$ and  $remainingSumOfSquare \gets n(n+1)(2n+1)/6$
      \State Process( $a_i $ ):
      		\Indent
     			\State $remainingSum := remainingSum - j$
			\State $remainingSumOfSquare := remainingSumOfSquare - j^2$
		\EndIndent
	\State Compute 2 missing items from $\frac{ remainingSum \pm \sqrt{ remainingSum^2 - 2( remainingSum^2 - remainingSumOfSquare ) } }{2}$\\
  \Return the missing items
\end{algorithmic}
\end{algorithm}

We will prove the correctness of the algorithm by analysing a quadratic equation degree 2. 
$$ (x+y)^2 = x^2 + 2xy + y^2 $$

Let $x$, $y$ denote 2 missing items in the stream and $\Delta_s$, $\Delta_{ss}$ denote $remainingSum$ and $remainingSumOfSquare$ after the algorithm processes $n-2$ tokens. Thus
\begin{align*}
	\Delta_s^2 &= \Delta_{ss} + 2xy \\
	2xy &= \Delta_s^2 - \Delta_{ss} \\
	xy &= \frac{\Delta_s^2 - \Delta_{ss}}{2}
\end{align*}

Next, we first simplify the equation by denoting $A$ as $(\Delta_s^2 - \Delta_{ss})/2$. We know that
\begin{align*}
	x+y &= \Delta_s \\
	x + A/x &= \Delta_s \\
	x^2+A  &= \Delta_s x\\
	x^2 - \Delta_s x + A &= 0 \\
	x &= \frac{\Delta_s + \sqrt{\Delta_s^2 - 4A}}{2} \\
	&= \frac{\Delta_s + \sqrt{\Delta_s^2 - 2(\Delta_s^2 - \Delta_{ss})}}{2}\\
\end{align*}
Thus
$$ 	y = \frac{\Delta_s - \sqrt{\Delta_s^2 - 2(\Delta_s^2 - \Delta_{ss})}}{2} $$
Therefore, the algorithm always returns correct answer. Next, considering bits of storage, we see that the algorithm need to store only 2 values, namely $totalSum$ and $totalSumOfSquare$ and the maximum value is $O(n^3)$. Thus, the algorithm need $O(\log n )$ bits to store such values.
